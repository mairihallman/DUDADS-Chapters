\documentclass{article}
\usepackage{graphicx} % Required for inserting images
\usepackage{amsmath} % For mathematical symbols and environments

\title{Deep Learning}
\author{Mairi Hallman}
\date{May 2024}

\begin{document}

\maketitle

\section{Introduction}

\section{A Brief Overview of Tensors}

You are likely familiar with scalars, vectors, and matrices. These can be thought of as analagous data structures in zero, one, and two-dimensions, respectively. When generalizing to \(N\) dimensions, we refer to these collectively as tensors. A scalar is a zero-order tensor, a vector is a first-order tensor, and a matrix is a second-order tensor. A third-order tensor can be visualized as a stack of matrices. A fourth-order tensor would then be a vector of third order tensors. A fifth-order tensor is a matrix of third-order tensors... and so on.

% notation

% fiber and slices

\subsection{Tensor Products}

Tensor additon and subtraction are self-explanatory if matrix addition and subtraction are understood. The same cannot be said for tensor products. Below is an overview of tensor products necessary for the decompositions that will be presented in the next section.

\subsubsection{Outer Product \(\circ\)}

A tensor \(T^{(N)}\) can be expressed as a product of \(N\) vectors. This is called the outer product (denoted \(\circ\)).

\[T^{(N)} = u_1 \circ u_2 \circ \dots \circ u_N\]

% make less hand-wavey

\subsubsection{Kronecker Product \(\otimes\)}

The Konecker product of two matrices \(A\) and B, their Kronecker product is a matrix of the products of each element in \(A\) and the entire matrix \(B\).

\[A \otimes B = \begin{bmatrix}
    a_{11}B & a_{12}B & \dots & a_{1n}B \\
    a_{21}B & a_{22}B & \dots & a_{2n}B \\
    \vdots & \vdots & \ddots & \vdots \\
    a_{m1}B & a{m2}B & \dots & a_{mn}B
\end{bmatrix} \]

\subsubsection{Khatri-Rao Product \(\odot\)}

The Khatri-Rao product of two matrices \(A\) and \(B\), each with the same number of columns, is a matrix composed of the Kronecker products of the columns in matrix \(A\) and the columns in matrix \(B\) with the same indices.

\[A^{:\times n} \odot B^{:\times n} = \begin{bmatrix}
    a_{:,1} \otimes b_{:,1} & a_{:,2} \otimes b_{:,2} & \dots & a_{:,n} \otimes b_{:,n}
\end{bmatrix} \]

\subsubsection{Hadamard Product \(*\)}

The Hadamard product of two matrices \(A\) and \(B\) of the same dimensions is a matrix formed of the products of the elements in \(A\) and \(B\) with the same indices.

\[A^{m \times n} * B^{m \times n} = \begin{bmatrix}
    a_{11}b_{11} & a_{12}b_{12} & \dots & a_{1n}b_{1n} \\
    a_{21}b_{21} & a_{22}b_{22} & \dots & a_{2n}b_{2n} \\
    \vdots & \vdots & \ddots & \vdots \\
    a_{m1}b_{m1} & a_{m2}b_{m2} & \dots & a_{mn}b_{mn}
\end{bmatrix} \]

\subsection{Tensor Decompositions}

\subsubsection{CP Decomposition}

\subsubsection{Tucker Decomposition}

\subsubsection{Tensor Train}

\section{Selecting A Network Architecture}

\section{Convolutional Neural Networks}

\subsection{What Is Convolution?}

% discrete case: \((a * b)_n = \sum\limits_{\substack{i,j \\ i+j=n}} a_i \cdot b_j\)

\subsection{Image Classification Example}

\section{Recurrent Neural Networks}

\section{Generative Models}

\subsection{Generative Adversarial Neural Networks}

\subsection{Variational Autoencoders}

\subsection{Quantization}

% can quantize weights or biases

% linear quantization ( see L4 quantization theory)

% example - 32-bit to 8-bit

% quanto library


\end{document}