\documentclass{article}
\usepackage{graphicx} % Required for inserting images
\usepackage{amsmath} % For mathematical symbols and environments

\title{Deep Learning}
\author{Mairi Hallman}
\date{May 2024}

\begin{document}

\maketitle

\section{Introduction}

\section{A Brief Overview of Tensors}

You are likely familiar with scalars, vectors, and matrices. These can be thought of as analagous data structures in zero, one, and two-dimensions, respectively. When generalizing to \(N\) dimensions, we refer to these collectively as tensors. A scalar is a zero-order tensor, a vector is a first-order tensor, and a matrix is a second-order tensor. A third-order tensor can be visualized as a stack of matrices. A fourth-order tensor would then be a vector of third order tensors. A fifth-order tensor is a matrix of third-order tensors... and so on.

% notation

% fiber and slices

\subsection{Tensor Products}

Tensor additon and subtraction are self-explanatory if matrix addition and subtraction are understood. The same cannot be said for tensor products. Below is an overview of important tensor products.

% need to figure out which tensor products to include

\subsection{Tensor Decompositions}

\subsubsection{CP Decomposition}

\subsubsection{Tucker Decomposition}

\subsubsection{Tensor Train}

\section{Selecting A Network Architecture}

\section{Convolutional Neural Networks}

\subsection{What Is Convolution?}

% discrete case: \((a * b)_n = \sum\limits_{\substack{i,j \\ i+j=n}} a_i \cdot b_j\)

\subsection{Image Classification Example}

\section{Recurrent Neural Networks}

\section{Generative Models}

\subsection{Generative Adversarial Neural Networks}

\subsection{Variational Autoencoders}


\end{document}